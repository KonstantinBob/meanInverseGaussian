\documentclass{scrreprt}
%\usepackage[ngerman]{babel}
\usepackage{ucs}
\usepackage[utf8]{inputenc}
\usepackage{enumitem}
\usepackage{amsfonts}
\usepackage{amsmath}
\usepackage{amssymb}
\usepackage{amsthm}
\usepackage{color}
\usepackage{graphicx}
\usepackage{graphics}
\usepackage{listings} \lstset{numbers=left, numberstyle=\tiny, numbersep=5pt} \lstset{language=C++} 
\setlength{\parindent}{0em}
\setlength{\parskip}{2.0ex plus 1.0ex minus 0.5ex}
\usepackage{geometry}%[a4paper,text={160mm,255mm},centering,headsep=5mm,footskip=10mm]
\geometry{a4paper,left=35mm,right=35mm, top=33mm, bottom=66mm}
\renewcommand*\chapterheadstartvskip{\vspace*{-6ex}}
\setkomafont{disposition}{\normalcolor\bfseries}
\setkomafont{descriptionlabel}{\normalcolor\bfseries}
\renewcommand{\d}{{\mathrm{d}}}
\newcommand{\com}[1]{}
\begin{document}%~~~~~~~~~~~~~~~~~~~~~~~~~~~~~~~~~~~~~~~~~~~~~~~~~~~~~~~~~
\chapter*{Computation of the mean of an inverse Gaussian distribution}
Consider an inverse Gaussian distribution with location parameter $mu$ and scale parameter $\lambda$. The pdf is given as
\begin{equation*}
 f(x;\mu ,\lambda )= \sqrt{\frac{\lambda }{2\pi x^{3}}} \exp \left( \frac{\lambda (x-\mu )^{2}}{2\mu ^{2}x} \right).
\end{equation*}
We are interested in computing the mean, i.e.
\begin{equation*}
\mathrm{E}[x] = \int_{0}^{\infty}\d x\,x\sqrt{\frac{\lambda }{2\pi x^{3}}} \exp \left( \frac{\lambda (x-\mu )^{2}}{2\mu ^{2}x} \right).
\end{equation*}
In order to do that consider $t > 0$ and define a function $g$
\begin{equation*}
g(t) = \int_{0}^{\infty}\d x\sqrt{\frac{\lambda }{2\pi x^{3}}} \exp \left( \frac{\lambda (tx-\mu )^{2}}{2\mu ^{2}x} \right).
\end{equation*}
The integral can be solved by the substitution $tx \rightarrow z$:
\begin{align*}
g(t)&=\int_{0}^{\infty}\d z\frac{1}{t}\sqrt{\frac{\lambda t^3}{2\pi x^{3}}} \exp \left( \frac{\lambda t(z-\mu )^{2}}{2\mu ^{2}z} \right)\\
&=\int_{0}^{\infty}\d z\sqrt{\frac{\lambda t}{2\pi x^{3}}} \exp \left( \frac{\lambda t(z-\mu )^{2}}{2\mu ^{2}z} \right)\\
&=1
\end{align*}
In the last line we used that the integral is the normalization of an inverse Gaussian distribution with location parameter $mu$ and scale parameter $\lambda t$.

Now we compute 
\begin{align*}
\frac{\partial g(t)}{\partial t}&=\int_{0}^{\infty}\d x\sqrt{\frac{\lambda }{2\pi x^{3}}} \exp \left( \frac{\lambda (tx-\mu )^{2}}{2\mu ^{2}x} \right)\left(- \lambda\frac{tx-\mu}{2 \mu^2 x}x\right)\\
&=-\frac{\lambda}{2 \mu^2}\left(t\mathrm{E}[x]-\mu\right)\\
\end{align*}

If we compare that with above, where $g(t) = 1 $ and thus  $\frac{\partial g(t)}{\partial t}=0$, we find
\begin{equation*}
t\mathrm{E}[x]-\mu = 0.
\end{equation*}
By setting $t = 1$ we obtain
\begin{equation*}
\mathrm{E}[x] = \mu.
\end{equation*}
\end{document}%~~~~~~~~~~~~~~~~~~~~~~~~~~~~~~~~~~~~~~~~~~~~~~~~~~~~~~~~~~~
